%%%%%%%%%%%%%%%%%
% This is an sample CV template created using altacv.cls
% (v1.7, 9 August 2023) written by LianTze Lim (liantze@gmail.com). Compiles with pdfLaTeX, XeLaTeX and LuaLaTeX.
%
%% It may be distributed and/or modified under the
%% conditions of the LaTeX Project Public License, either version 1.3
%% of this license or (at your option) any later version.
%% The latest version of this license is in
%%    http://www.latex-project.org/lppl.txt
%% and version 1.3 or later is part of all distributions of LaTeX
%% version 2003/12/01 or later.
%%%%%%%%%%%%%%%%

%% Use the "normalphoto" option if you want a normal photo instead of cropped to a circle
% \documentclass[10pt,a4paper,normalphoto]{altacv}

\documentclass[10pt,a4paper,ragged2e,withhyper]{altacv}
%% AltaCV uses the fontawesome5 and packages.
%% See http://texdoc.net/pkg/fontawesome5 for full list of symbols.

% Change the page layout if you need to
\geometry{left=1.25cm,right=1.25cm,top=1.5cm,bottom=1.5cm,columnsep=1.2cm}

% The paracol package lets you typeset columns of text in parallel
\usepackage{paracol}

% Change the font if you want to, depending on whether
% you're using pdflatex or xelatex/lualatex
% WHEN COMPILING WITH XELATEX PLEASE USE
% xelatex -shell-escape -output-driver="xdvipdfmx -z 0" sample.tex
\ifxetexorluatex
  % If using xelatex or lualatex:
  \setmainfont{Roboto Slab}
  \setsansfont{Lato}
  \renewcommand{\familydefault}{\sfdefault}
\else
  % If using pdflatex:
  \usepackage[rm]{roboto}
  \usepackage[defaultsans]{lato}
  % \usepackage{sourcesanspro}
  \renewcommand{\familydefault}{\sfdefault}
\fi



% Change the colours if you want to
\definecolor{SlateGrey}{HTML}{2E2E2E}
\definecolor{LightGrey}{HTML}{666666}
\definecolor{DarkPastelRed}{HTML}{450808}
\definecolor{PastelRed}{HTML}{8F0D0D}
\definecolor{GoldenEarth}{HTML}{E7D192}
\colorlet{name}{black}
\colorlet{tagline}{PastelRed}
\colorlet{heading}{DarkPastelRed}
\colorlet{headingrule}{GoldenEarth}
\colorlet{subheading}{PastelRed}
\colorlet{accent}{PastelRed}
\colorlet{emphasis}{SlateGrey}
\colorlet{body}{LightGrey}

% Change some fonts, if necessary
\renewcommand{\namefont}{\Huge\rmfamily\bfseries}
\renewcommand{\personalinfofont}{\footnotesize}
\renewcommand{\cvsectionfont}{\LARGE\rmfamily\bfseries}
\renewcommand{\cvsubsectionfont}{\large\bfseries}


% Change the bullets for itemize and rating marker
% for \cvskill if you want to
\renewcommand{\cvItemMarker}{{\small\textbullet}}
\renewcommand{\cvRatingMarker}{\faCircle}
% ...and the markers for the date/location for \cvevent
% \renewcommand{\cvDateMarker}{\faCalendar*[regular]}
% \renewcommand{\cvLocationMarker}{\faMapMarker*}


% If your CV/résumé is in a language other than English,
% then you probably want to change these so that when you
% copy-paste from the PDF or run pdftotext, the location
% and date marker icons for \cvevent will paste as correct
% translations. For example Spanish:
% \renewcommand{\locationname}{Ubicación}
% \renewcommand{\datename}{Fecha}


%% Use (and optionally edit if necessary) this .tex if you
%% want to use an author-year reference style like APA(6)
%% for your publication list
% % When using APA6 if you need more author names to be listed
% because you're e.g. the 12th author, add apamaxprtauth=12
\usepackage[backend=biber,style=apa6,sorting=ydnt]{biblatex}
\defbibheading{pubtype}{\cvsubsection{#1}}
\renewcommand{\bibsetup}{\vspace*{-\baselineskip}}
\AtEveryBibitem{%
  \makebox[\bibhang][l]{\itemmarker}%
  \iffieldundef{doi}{}{\clearfield{url}}%
}
\setlength{\bibitemsep}{0.25\baselineskip}
\setlength{\bibhang}{1.25em}


%% Use (and optionally edit if necessary) this .tex if you
%% want an originally numerical reference style like IEEE
%% for your publication list
\usepackage[backend=biber,style=ieee,sorting=ydnt,defernumbers=true]{biblatex}
%% For removing numbering entirely when using a numeric style
\setlength{\bibhang}{1.25em}
\DeclareFieldFormat{labelnumberwidth}{\makebox[\bibhang][l]{\itemmarker}}
\setlength{\biblabelsep}{0pt}
\defbibheading{pubtype}{\cvsubsection{#1}}
\renewcommand{\bibsetup}{\vspace*{-\baselineskip}}
\AtEveryBibitem{%
  \iffieldundef{doi}{}{\clearfield{url}}%
}



\begin{document}
\name{Willian Carlos Agostini}
\tagline{Software Developer}
%% You can add multiple photos on the left or right
% \photoR{2.8cm}{Globe_High}
% \photoL{2.5cm}{Yacht_High,Suitcase_High}

\personalinfo{%
  % Not all of these are required!
  \email{willian.agostini@gmail.com}
  \phone{(54) 99985-4701}
  \location{Três Palmeiras, RS}
  \linkedin{agostini-willian}
  \github{WillianAgostini}
  %% You can add your own arbitrary detail with
  %% \printinfo{symbol}{detail}[optional hyperlink prefix]
  % \printinfo{\faPaw}{Hey ho!}[https://example.com/]

  %% Or you can declare your own field with
  %% \NewInfoFiled{fieldname}{symbol}[optional hyperlink prefix] and use it:
  % \NewInfoField{gitlab}{\faGitlab}[https://gitlab.com/]
  % \gitlab{your_id}
  %%
  %% For services and platforms like Mastodon where there isn't a
  %% straightforward relation between the user ID/nickname and the hyperlink,
  %% you can use \printinfo directly e.g.
  % \printinfo{\faMastodon}{@username@instace}[https://instance.url/@username]
  %% But if you absolutely want to create new dedicated info fields for
  %% such platforms, then use \NewInfoField* with a star:
  % \NewInfoField*{mastodon}{\faMastodon}
  %% then you can use \mastodon, with TWO arguments where the 2nd argument is
  %% the full hyperlink.
  % \mastodon{@username@instance}{https://instance.url/@username}
}

\makecvheader
%% Depending on your tastes, you may want to make fonts of itemize environments slightly smaller
% \AtBeginEnvironment{itemize}{\small}

%% Set the left/right column width ratio to 6:4.
\columnratio{0.6}

% Start a 2-column paracol. Both the left and right columns will automatically
% break across pages if things get too long.
\begin{paracol}{2}

\cvsection{Projetos}

\cvevent{Axios}{Contribuidor no projeto open source github/axios}{Mar 2022 - o momento}{}
Contribuições que incluíram a resolução de bugs, a adição de novos recursos e a melhoria da documentação do repositório.


Confira minhas contribuições nesse projeto \href{https://github.com/axios/axios/issues?q=commenter%3AWillianAgostini}{\textcolor{blue}{\underline{github/axios}}}.

\cvsection{Experiência}

\cvevent{Senior Backend Developer}{Compass.uol}{Jul 2022 - o momento}{Remoto}

\begin{itemize}
  \item Especialista em Node.js, utilizando JavaScript, TypeScript, AWS, Jest e NestJS para criar soluções escaláveis.
  \item Implementação de padrões de segurança, otimização de performance e automação de testes em aplicações web.
  \item Consultor na Fleetcor Technologies, líder global em pagamentos corporativos, atuando no time de desenvolvimento do SuperApp Sem Parar.
\end{itemize}

\divider

\cvevent{Full Stack Developer}{Trafegus Sistemas}{Nov 2020 -- Jul 2022}{Chapecó, SC}
\begin{itemize}
  \item Desenvolvimento de aplicações utilizando Node.js, JavaScript, Ionic, PHP, Jest, Python e Docker.
  \item Construção e manutenção de aplicações web de alta performance e escaláveis.
\end{itemize}

\divider

\cvevent{Full Stack Developer}{Bionexus Allagro}{Mar 2019 -- Nov 2020}{Chapecó, SC}
\begin{itemize}
  \item Reestruturação da arquitetura de software para reduzir o consumo de recursos.
  \item Trabalhei como desenvolvedor web utilizando Python, Docker e Angular.
  \item Desenvolvimento de aplicativos com Ionic.
\end{itemize}

\divider
\cvevent{Full Stack Developer}{M8 Sistemas}{Jan 2017 -- Nov 2017}{Chapecó, SC}
\begin{itemize}
  \item Projeto e desenvolvimento de aplicativos móveis e produtos web.
\end{itemize}


\medskip

% use ONLY \newpage if you want to force a page break for
% ONLY the current column

\cvsection{Educação}

\cvevent{Pós-graduação Lato Sensu - Especialização em Inteligência Artificial}{PUC Minas}{Fev 2021 -- Dez 2022}{}

\divider

\cvevent{Bacharelado em Engenharia de Computação}{Universidade do Oeste de Santa Catarina}{Fev 2013 -- Jun 2020}{}

\medskip

% use ONLY \newpage if you want to force a page break for
% ONLY the current column

\cvsection{Certificações}

\begin{itemize}
  \item Desenvolvimento AWS 2020 com foco em Serverless;
  \item Padrões de Projeto GoF (Design Patterns);
  \item Node JS: Advanced Concepts;
  \item Mastering ChatGPT Models: From Fine-tuning to Deployment;
  \item Advanced SQL;
\end{itemize}

O restante das certificações e seus certificados podem ser encontradas \href{https://www.linkedin.com/in/agostini-willian/details/certifications/}{\textcolor{blue}{\underline{aqui}}}.

\cvsection{Demais projetos}
Confira minhas contribuições nesses projetos 

\href{https://github.com/Unitech/pm2/issues?q=commenter%3AWillianAgostini+}{\textcolor{blue}{\underline{Unitech/pm2}}}

\href{https://github.com/jestjs/jest/issues?q=commenter%3AWillianAgostini+}{\textcolor{blue}{\underline{jest/jest}}}

\href{https://github.com/sindresorhus/got/issues?q=commenter%3AWillianAgostini+}{\textcolor{blue}{\underline{sindresorhus/got}}}

\href{https://github.com/BrasilAPI/BrasilAPI/issues?q=commenter%3AWillianAgostini+}{\textcolor{blue}{\underline{BrasilAPI/BrasilAPI}}}

\href{https://github.com/WillianAgostini/brasilapi-js}{\textcolor{blue}{\underline{WillianAgostini/brasilapi-js}}}



%% Switch to the right column. This will now automatically move to the second
%% page if the content is too long.
\switchcolumn

\cvsection{habilidades interpessoais}

\cvtag{Autodidatismo} 
\cvtag{Adaptabilidade}
\cvtag{Respeito}
\cvtag{Colaboração / Trabalho em equipe}
\cvtag{Resiliência}
\cvtag{Organização}

\cvsection{habilidades técnicas}

\begin{itemize}
\item{Principais Linguagens de Desenvolvimento}
\end{itemize}

\cvtag{Javascript}
\cvtag{Typescript}
\cvtag{Python}

\divider\smallskip

\begin{itemize}
\item{Principais Frameworks e Plataformas}
\end{itemize}
\cvtag{Node.js}
\cvtag{Jest}
\cvtag{Axios}
\cvtag{NestJS}
\cvtag{Express}

\divider\smallskip

\begin{itemize}
\item{Nuvem}
\end{itemize}
\cvtag{AWS}
\cvtag{AWS Lambda}
\cvtag{Api Gateway}
\cvtag{SQS}
\cvtag{S3}

\divider\smallskip

\begin{itemize}
\item{Banco de Dados}
\end{itemize}
\cvtag{PostgreSQL}
\cvtag{DynamoDB}

\divider\smallskip

\begin{itemize}
\item{Princípios de Design e Desenvolvimento}
\end{itemize}
\cvtag{SOLID}
\cvtag{KISS}
\cvtag{DRY}

\divider\smallskip

\begin{itemize}
\item{Hospedagem de Repositório, CI/CD}
\end{itemize}
\cvtag{GitLab}
\cvtag{GitHub}
\cvtag{AzureDevops}

\divider\smallskip

\begin{itemize}
\item{Apis} 
\end{itemize}
\cvtag{Rest}
\cvtag{GraphQL}
\cvtag{Soap}
\cvtag{OAuth}
\cvtag{JWT}

\divider\smallskip

\begin{itemize}
\item{Testes}
\end{itemize}
\cvtag{Unidade}
\cvtag{E2E}

\cvsection{Linguagem}

\cvskill{English}{3}
%% Supports X.5 values.

%% Yeah I didn't spend too much time making all the
%% spacing consistent... sorry. Use \smallskip, \medskip,
%% \bigskip, \vspace etc to make adjustments.
\medskip




\end{paracol}


\end{document}
